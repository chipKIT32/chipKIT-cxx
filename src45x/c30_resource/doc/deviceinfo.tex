\documentclass{article}
\usepackage{alltt,changebar,longtable}
\begin{document}
\begin{titlepage}
\begin{center}
\LARGE{Design: C30 Device Information Resource} \\
\vspace{10 mm}
Requirements and Interface Specification \\
\vspace{5 mm}
\begin{small}
\begin{tabular}{|l|l|p{7cm}|l|} \hline
Version&Author&Comments&Date\\
\hline
Device A&C. Wilkie& Initial& 16 Feb 2006\\
Device B&C. Wilkie& Incorporate high level design& 28 Feb 2006\\
Device C&C. Wilkie& Incorporate high level design commnets& 6 Mar 2006\\
Device D&C. Wilkie& Additions since design& 21 Mar 2006\\
\hline
\end{tabular}
\newline
\vspace{50mm}
\begin{tabular}{|l|l|l|} \hline
  Approval & Signature & Date\\
\hline
  Mike Collison & \hspace{55mm} & \hspace{20mm} \\
  \hspace{40mm} & \hspace{55mm} & \hspace{20mm} \\
\hline
  Guy McCarthy & \hspace{55mm} & \hspace{20mm} \\
  \hspace{40mm} & \hspace{55mm} & \hspace{20mm} \\
\hline
  Calum Wilkie & \hspace{55mm} & \hspace{20mm} \\
  \hspace{40mm} & \hspace{55mm} & \hspace{20mm} \\
\hline
  Jerry Nairn & \hspace{55mm} & \hspace{20mm} \\
  \hspace{40mm} & \hspace{55mm} & \hspace{20mm} \\
\hline
\end{tabular}

\end{small}
\end{center}
\end{titlepage}
\tableofcontents
\pagebreak

\section{Overview}

  The purpose of this development activity is to reduce the requirement to
release new binaries to support new devices.  We plan to de-couple the
device specific information from the executable files and place them into
a resource file.  This will allow us to provide new device support simply 
by providing the external support files: linker scripts, assembler include 
files, C header files, and libraries along with a new resource file.

  The will allow the student edition to continue to support new devices
without requiring a re-build and underscore the philosophy that improved
functionality is only made available to paying customers (and to the
student edition in greater intervals).

\section{High Level Requirements}

  The feature must:
  \begin{itemize}
    \item support upgrading without rebuilding the binary tools
    \item provide supplementary version information
    \item encapsulate the device characteristics required by the tool chain
    \item be robust enough that users cannot break it
  \end{itemize}

\subsection{Upgrade paths}

  The feature will initially be released with modified executable files along 
with a resource file.  Future releases may only require support files to be 
released; this will be handled with a reduced installer.

\subsection{Version information}

  The resource file will contain, along with other information, a version
identifier which the tool chain will report along with the binary's version
string.

  For example: 
  \begin{enumerate} 
    \item An initial release will co-incide with version 2.02. 
          The resource's version increment would be \texttt{A}.
          The tool chain would report version: \texttt{v2.02 (A)}.  

    \item A future update may be produced where we would only release
          new device support, and not rebuild the binaries.  We release
          an update file providing the latest support files and resource
          file (increment \texttt{B}).  After installation of the update file
          the tool chain would report version: \texttt{v2.02 (B)}.  Which 
          signifies that the binaries are still those of v2.02, but the 
          support files have been revised.

    \item It may be case that a customer wishes to use a resource file
          that was not released with their current version.  In that case,
\chgbarbegin
          the a binary in MPLAB C30 would report a different style of version
           number.
\chgbarend
          For example, the binaries version is v3.00 and the resource 
          increment \texttt{B}, released for version v2.02:
          \texttt{v3.00 resource version v2.02 (B)}.
  \end{enumerate}

\chgbarbegin
  The version of the resource file will \textit{always} correspond to the
  latest full binary release available.  A student edition compiler using the
  latest support files would almost certainly display the lengthened form of
  the version number.
\chgbarend
 

\subsection{Device characteristics}
 
  The resource file will provide the information for each tool that is currently
  provided as part of the binary.
  
  The compiler requires the following information:  string identifying the
  device name and an identification of which family to which the device 
  belongs.  Currently supported families are: 30FXXXX, 30F202X (or SMPS),
  24F, 24H, 33F, and a generic part.  A new device family would likely require
  additional work within the executable files.

  The binary utilities (\texttt{pic30-as}, \texttt{pic30-ld}, etc) require the 
  following information:  string identifying the device name, an integer 
  representing the processor ID which gets placed in the object/ executable 
  file, flags describing the presence/ absence of features such as: DSP, 
  EEDATA, SMPS, DMA, and a device family.

  There is a considerable amount of overlap between the two requirements.

\subsection{Robustness}

  The resource file will ideally be encoded in such a way such that the 
contents are not obvious to the the casual reader.  We will create a binary
file and a tool to create the file from a more reasonable source.

\section{Design considerations}

  In addition to the high level requirements presented earlier, there are
other design considerations.  Additional user considerations:

  \begin{itemize}
    \item the resource file format must be extensible
    \item later versions of the binaries must be able to use older versions
          of the resource file
    \item earlier versions of the binaries must be able to use later versions
          of the resource file
  \end{itemize}

  These may seem like bizarre requests to make but we have often seen cases
\chgbarbegin
where users only want to update MPLAB C30 in limited circumstances, for example,
\chgbarend
to obtain support for a new device or to update the binaries but keep the
support for an older device (which may have disappeared).  In such cases
the tool chain should issue a warning stating that that the resource version
does not match the current tool chain.  Caveat emptor.

  Additional implementation considerations:

  \begin{itemize}
    \item the resource file should be generated from a currently known source
          of information (such as the DVS file)
    \item future extensions to the contents of the resource file should be
          made in such a way that they do not conflict with any current
          implementation  (fields should not be deleted or have their semantics
          changed)
    \item future extensions to the tools that read the resource file should
          behave appropriately when presented with a resource file that
          does not contain the additional information; examples of appropriate
          behaviour include: generating an error message and stopping, 
          making a default assumption, or continuing without using the
          (missing) additional information.  Whatever seems appropriate.
\chgbarbegin
    \item Student Editions of tool chains may choose to care less about 
          matching version identifiers and omit producing a warning if the 
          resource version mis-matches the current student edition.  This will 
          certainly be the case for MPLAB C30, which will only emit a warning 
          if the major version numbers mis-match.
\chgbarend
  \end{itemize}

\section{Implementation Details}

\subsection{Release information}

\chgbarbegin
  Support update releases will be delivered in one of three ways, depending 
\chgbarend
  upon circumstance.

  \begin{description}
  \item [full release]    where the binaries have been modified in some way, 
                          will be released in a manner consistent with today.  
                          An installer package will be created which contains: 
                          the tool chain binaries, support files, library files,
                          examples, and so on.  The installer will be similar
                          to those that currently exist (with the addition of
                          the resource file).  The download page link would
                          be titled as today: \textbf{MPLAB C30 v$X.YY$}, it
                          would replace any previous link for a full release. 
                          Any \textbf{Support update $\ldots$} links will be
                          removed. A student edition installer may or may not
                          be created.

  \item [update release]  where the binaries have not been modified, but new
                          support files are required.  An installer package
                          will be created containing all current (not just
                          new files) versions of the support files and library
                          files.  The installer will be similar to those that
                          currently exist, with the following changes:
                          \begin{itemize}
                             \item the binary files will not be updated
                             \item the installer will warn the user if an 
                                   attempt to install resource files designed
                                   for another version is made
                          \end{itemize}
                          The download page link would be entitled:
                          \textbf{Support update $A$ for version $X.YY$}.
                          This link would replace any previous \textbf{Support 
                          update $\ldots$} link.

  \item [student release] where a \textit{full release} is made but for
                          which we do not wish to upgrade the compiler 
                          available with the student edition.  This will be a 
                          separate installer much like the installer for
                          an \textit{update release}.  
                          The download page link would be entitled:
                          \textbf{Support update $A$ for version $X.YY$}.
                          This link would replace any previous \textbf{Support 
                          update $\ldots$} link.  Note that the student
                          edition version may be different than the current
                          full release version.
  \end{description}

\subsection{Resource file format}

  The resource file format will be a binary encoded format which is processed
by a utility from a textual input language.  The input will be derived from
the DVS input sources.  For devices which do not currently have a DVS file
specification, the resource input will be hand crafted.

\chgbarbegin
  The processor id is not currently defined inside a DVS file, this will need
to be acquired from another source.  The same export script (that parses the
DVS file) will also be able to query the processor ID database to acquire
the correct fields.
\chgbarend

\subsubsection{Textual resource format}

  The input language is straight forward and will consist of an 
\textit{introduction block} and a \textit{resource block}.  
The \textit{introduction block} will describe version numbers and the further 
file content in the \textit{resource block}.  The resource input file will
be pre-processed first, so any C pre-processor directive can be used.

  The language will accept:
  \begin{itemize}
    \item string literals, delineated by quotation marks: \texttt{"foo"}
    \item character literals, delineated by single quote marks: \texttt{'A'}
    \item version numbers, 16 bit literal.16 bit literal: \texttt{2.02}
    \item a value array, delineated by curly braces
    \item 32-bit integer literals, in hex or decimal notation: \texttt{0x1234}
    \item simple mathematical operations on integer literals: +,-,$\star$,
          /,\&,$\mid$,$\wedge$
    \item \# will introduce a comment to the end of line
  \end{itemize}

  The following assignments must be present (in the given order) in the 
  introduction block, examples given are for the first release (with 
  fictitious data):
\chgbarbegin
  \begin{verbatim}
     tool_chain = "C30"           # identifies a resource file
                                  # for C30
     tool_version = 3.02          # this file is released for 
                                  # v3.02
     resource_version = 'A'       # resource increment version
                                  # for v3.02
     field_count = 3              # defines the number of fields 
                                  # per entry
     field_sizes = { 10, 4, 4 }   # number of bytes for each 
                                  # field in an entry
     resource_info_start          # marks the end of the 
                                  # information block
  \end{verbatim}

  Note that the \texttt{resource\_info\_start} marker is represented in the
  resource file as a fixed sequence of bytes: \texttt{0xFF 0xFF 0xFF 0xFF}
  so this sequence should be avoided in the resource information block.

  It is anticipated that future extensions will allow target specific data
  to be easily defined in the resource information block.
\chgbarend
  
  Once the \texttt{resource\_info\_start} token has been reached, the format
  will be more free form and consist of an array of literals, one for each
  field, separated by commas.  Follows is a fictitious example for C30.
  \begin{verbatim}
     { "30F6014", 0x1234, 1 }
     { "30F6014A", 0x1234, 2 }
  \end{verbatim}

\chgbarbegin
  The meaning of each field is not interpreted by the resource generation
  tool.  
\chgbarend
  The generation tool will give various errors: 
  \begin{itemize}
     \item if the field value exceeds the desired field size
     \item if there are insufficient fields values provided
     \item if there are too many field values provided
  \end{itemize}
  
\subsubsection{Binary resource format}

  The binary format will very closely match that of the text version.  However,
  tag names will not be encoded and literal string and character values will
  be encoded to obfuscate their values.

  Each character will have the following transformation applied to it:
  \begin{verbatim}
    c = (((c << 1) & 0xAA) | 
         ((c>>1) & 0x55));    /* swap every bit */
    c = (c >> 4) | (c << 4);  /* swap every nybble */
  \end{verbatim}

  This will convert "30F6014" (0x33 0x30 0x46 0x36 0x30 0x31 0x34) into
  gibberish (though, the '3' remains) the hex would be 0x33 0x03 0x98 
  0x93 0x03 0x23 0x83.

\chgbarbegin
  Short fields will be padded with 0's (after the data) to fill up the 
  required space, as defined by the field width in the resource information
  block.
\chgbarend
  All literals (numbers and stings) will be placed LS byte first.  The least
  significant byte of a string is on the right hand side, the null termination
  remains at the end.

  The complete binary representation of the file given previously would be 
  (spaces, newlines, and comments to aid read-ability):
  \begin{verbatim}
     0x03 0x33 0x38 0x00    # "C30"
     0x03 0x00 0x02 0x00    # 3.02
     0x28                   # 'A'
     0x03 0x00 0x00 0x00    # 3
     0x0A 0x00 0x00 0x00    # 10
     0x04 0x00 0x00 0x00    # 4
     0x04 0x00 0x00 0x00    # 4
     0xFF 0xFF 0xFF 0xFF    # resource info start marker
     0x83 0x23 0x03 0x93 0x98 0x03 0x33 0x00 0x00 0x00
                            # "30F6014"
     0x34 0x12 0x00 0x00    # 0x1234
     0x01 0x00 0x00 0x00    # 1
     0x28 0x83 0x23 0x03 0x93 0x98 0x03 0x33 0x00 0x00
                            # "30F6014A"
     0x34 0x12 0x00 0x00    # 0x1234
     0x01 0x00 0x00 0x00    # 2
   \end{verbatim}

  The end of file signifies the end of the records.

\subsection{Tool chain modifications}

  \subsubsection{New features}

  There is a certain amount of shared code that can be utilized between
binutils and the compiler proper; it is up to the implementor to determine
how best to share this code (either through a library, a shared block of
source code, or another way yet to be determined).

  The following generic types will be created:

\begin{verbatim}
  struct resource_version {
    short major;
    short minor;
  };

  struct resource_introduction_block {
    char *tool_chain;
    char resource_version_increment;
    struct resource_version version;
    unsigned int field_count;
    unsigned int *field_sizes;
    void *target_specific_data;
  };

  enum resource_information_kind {
    rik_error = 0,
    rik_string,
    rik_char,
    rik_version,
    rik_int
  };

  struct resource_data {
    enum resource_information_kind kind;
    union {
      const char *s;
      char c;
      struct resource_version version;
      unsigned int i;
    } v;
  };
\end{verbatim}

  The \texttt{resource\_introduction\_block} may be extended by a particular
  implementation/ version as an explicit marker is placed between the 
  introduction block and the start of the resource data itself.

  The following functions will be provided: \\

  $\bullet$ \textbf{\texttt{struct resource\_introduction\_block $\star$read\_rib(const char $\star$name)}} \newline

  \begin{minipage}[c]{10cm}
  Open file \textit{name} and read the resource introduction defined by the
  standard \texttt{resource\_introduction\_block}.  \\

  This function will leave the \texttt{FILE $\star$} at the last read entry, 
  so that a target can read more entries if required (using 
  \texttt{read\_value}).  The \texttt{target\_specific\_data} field in a 
  \texttt{resource\_introduction\_block} can contain a pointer to a target
  specific structure where additional entries may be stored.  An implementation
  can validate the version of the resource file to determine if additional
  fields are present, and manually read them. \\

  Only one resource file may be open at a time.  The function will return 
  zero to signify an error.  \\
  \end{minipage}

  $\bullet$ \textbf{\texttt{void close\_rib(void)}} \newline

  \begin{minipage}[c]{10cm}
  Close any open resource file; no error is returned if there is none.  This
  will also free any memory claimed. \\
  \end{minipage}

  $\bullet$ \textbf{\texttt{int read\_value(enum resource\_information\_kind,}\\
     \hspace*{34mm}  \texttt{struct resource\_data $\star$data)}} \newline

  \begin{minipage}[c]{10cm}
  Read the number of bytes required for the specified kind advancing the 
  \texttt{FILE $\star$} appropriately.  \textit{data} is completed with the
  correct value and kind.  For types of \texttt{rik\_string}, the current
  field number is used to examine its width as defined in the resource 
  information block.  A \texttt{rik\_string} cannot
  be read unless we are in a valid resource record, so \texttt{read\_byte()}
  should be used when reading extra strings from the information block. \\

  Non-zero is returned to signify success. \\
  \end{minipage}

  $\bullet$ \textbf{\texttt{int read\_byte(char $\star$byte)}} \newline

  \begin{minipage}[c]{10cm}
  Just want a byte?  Use this function. \\
  \end{minipage}

  $\bullet$ \textbf{\texttt{int move\_to\_record(int n)}} \newline

  \begin{minipage}[c]{10cm}
    Place the \texttt{FILE $\star$} at the start of record \textit{n}, where
    0 represents the first record in the resource.  This will advance past
    the \texttt{resource\_info\_start} entry if required.  It is possible to
    rewind the file to a previous entry. \\
   
    Non-zero is returned to signify success in reading that record. \\
  \end{minipage}

  In general the correct flow for reading a resource block \textit{might} be:
  \begin{verbatim}
    struct resource_data data;

    if ((rib = read_rib("foo")) == 0) error();
    if (rib->version.major > n) {
      struct target_data *extra_data;

      extra_data = (struct target_data *)
                   malloc(sizeof(struct target_data));
      rib->target_specific_data = extra_data;
      read_value(rik_int, &data);
      extra_data->my_data = data.v.i;
      /* etc */
    }
    
    for (record = 0; move_to_record(record); record++) {
      /* based on field_count and/or field_sizes[n] 
         read the correct fields */
      if (rib->field_count >= 3) {
        /* read three fields I know about */
        read_value(rik_string, &data);
        /* and so on */
      }
    }
  \end{verbatim}

  \subsubsection{Code updates}

  Various modifications need to be made to the current source to make use
  of the new resource file.  Each tool (binutils or the compiler) will use
  a different subset (overlapping) of the data described in the resource file.

  For binutils, the resource file will replace a table that is defined in \\
  \texttt{bfd/pic30-procs.h}; which is used to create a static table in
  \texttt{bfd/cpu-pic30.c}.  This (static) table will need to be created at 
  run-time (there are various ways to make this happen, any number of which is
  acceptable and left up to the implementor).

  For GCC, the resource file will replace a table that is defined in \\
  \texttt{gcc/config/pic30/pic30.c}.  This (static) table is only used within \\
  \texttt{pic30\_override\_options()} to validate the \texttt{--mcpu} option.
  The table does not need to be reproduced, but the resource file can be
  scanned, instead, for an appropriate device name (or failure).  The
  precise implementation will be left up to the implementor.

\section{Implementation Details}

\subsection{v2.02}

  \subsubsection{resource file format}

  Each resource entry is equivalent to the following C structure:
  \begin{verbatim}
  struct resource_data {
    char id[29];
    unsigned int flags;
    unsigned int proc_id;
  };
  \end{verbatim}

  \begin{description}
    \item [\texttt{id}] a character string representing a device name or
                        a valid interrupt vector name

    \item [\texttt{flags}] a bit mask containing flags describing the id, with 
                           the following values
    \begin{description}
      \item [0x1] a normal dsPIC 30F device
      \item [0x2] a SMPS dsPIC 30F device
      \item [0x4] a dsPIC 33F device
      \item [0x8] a PIC 24F device
      \item [0x10] a PIC 24H device
      \item [0x100] the device has DSP support
      \item [0x200] the device has EEDATA support
      \item [0x400] the device has DMA support
      \item [0x20000000] the identifier is the name of a vector ID, if a
                         non-zero \texttt{proc\_id} is given the vector name is
                         only valid for that part, otherwise the device family
                         flags define the valid families
      \item [0x40000000] the identifier represents the name of a device
    \end{description}

    \item [\texttt{proc\_id}] the processor id for the named device, or the
                              processor id for which the vector name is valid
   \end{description}

\end{document}
